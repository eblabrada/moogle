\documentclass{article}
\usepackage[total={18cm,21cm},top=2cm, left=2cm]{geometry}
\usepackage{xcolor}
\usepackage{amsmath, amssymb, amsfonts}
\usepackage{url, hyperref}

%poner codigo fuente en latex
\usepackage{color}
\definecolor{gray97}{gray}{.97}
\definecolor{gray75}{gray}{.75}
\definecolor{gray45}{gray}{.45}

\usepackage{listings}
\lstset{ frame=Ltb,
	framerule=0pt,
	aboveskip=0.5cm,
	framextopmargin=2pt,
	framexbottommargin=2pt,
	framexleftmargin=0.4cm,
	framesep=0pt,
	rulesep=.4pt,
	backgroundcolor=\color{white},
	rulesepcolor=\color{black},
	%
	stringstyle=\ttfamily,
	showstringspaces = false,
	basicstyle=\small\ttfamily,
	commentstyle=\color{gray45},
	keywordstyle=\bfseries,
	%
	numbers=none,
	numbersep=15pt,
	numberstyle=\tiny,
	numberfirstline = false,
	breaklines=true,
}

\usepackage{listings}
\lstset{ frame=Ltb,
	framerule=0pt,
	aboveskip=0.5cm,
	framextopmargin=2pt,
	framexbottommargin=2pt,
	framexleftmargin=0.4cm,
	framesep=0pt,
	rulesep=.4pt,
	backgroundcolor=\color{white},
	rulesepcolor=\color{black},
	%
	stringstyle=\ttfamily,
	showstringspaces = false,
	basicstyle=\small\ttfamily,
	commentstyle=\color{gray45},
	keywordstyle=\bfseries,
	%
}

% minimizar fragmentado de listados
\lstnewenvironment{listing}[1][]
{\lstset{#1}\pagebreak[0]}{\pagebreak[0]}

\lstdefinestyle{consola}
{basicstyle=\scriptsize\bf\ttfamily}

\lstdefinestyle{}{language=c++,}

\begin{document}

\title{\bf \LARGE Moogle!}
\author{Eduardo Brito Labrada}
\date{\today}

\maketitle

\section{Una breve introducci\'on}

Moogle! es una aplicación {\it totalmente original} cuyo propósito es buscar inteligentemente un texto en un conjunto de documentos. Es una aplicación web, desarrollada con tecnología {\tt \color{gray45}.NET Core 6.0}, específicamente usando Blazor como {\it framework} web para la interfaz gráfica, y en el lenguaje {\tt \color{gray45} C\#}.
La aplicación está dividida en dos componentes fundamentales:

\begin{itemize}
	\item {\tt \color{gray45} MoogleServer} es un servidor web que renderiza la interfaz gráfica y sirve los resultados.
	\item {\tt \color{gray45} MoogleEngine} es una biblioteca de clases donde está\dots ehem\dots casi implementada la lógica del algoritmo de búsqueda.
\end{itemize}

\subsection{?`Para qu\'e sirve?}

La idea original del proyecto es buscar en un conjunto de archivos de texto (con extensión {\tt \color{gray45} .txt}) que estén en la carpeta {\tt \color{gray45}Content}.

\subsection{?`C\'omo usarlo?}

Primeramente, se aconseja a quien use esta aplicaci\'on tener instalado \href{https://es.wikipedia.org/wiki/Linux}{Linux}, ya que no se garantiza la misma eficiencia si esta en un dispositivo que use Windows.
En caso de tener instalado Windows, puede optar por \href{https://learn.microsoft.com/es-es/windows/wsl/install}{instalar Windows Subsystem for Linux (WSL)} que a\~nade
funcionalidades de Linux en Windows.

\subsubsection*{Instrucciones}

Lo primero que tendrás que hacer para poder trabajar en este proyecto es \href{https://learn.microsoft.com/es-es/dotnet/core/install/}{instalar {\tt \color{gray45}.NET Core 6.0}}.
Luego, te debes parar en la carpeta del proyecto y dependiendo de tu Sistema Operativo hacer lo siguiente:

\begin{itemize}
	\item {\bf Linux:} Debes tener instalado {\tt \color{gray45} make}. Si no lo tienes instalado
puedes instalarlo usando {\tt \color{gray45}sudo apt update \&\& sudo apt install make}. Luego podr\'as hacer {\tt \color{gray45}make dev}
	
	\item {\bf Windows:} Deberías poder ejecutar este proyecto usando {\tt \color{gray45}dotnet watch run --project MoogleServer}
\end{itemize}

Despu\'es de hacer lo anterior podr\'as abrir en tu navegador \href{http://localhost:5000}{http://localhost:5000} y podr\'as usar Moogle!

\end{document}
