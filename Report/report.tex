\documentclass{report}
\usepackage[total={18cm,21cm},top=2cm, left=2cm]{geometry}
\usepackage{xcolor}
\usepackage{amsmath, amssymb, amsfonts}
\usepackage{url, hyperref}

%poner codigo fuente en latex
\usepackage{color}
\definecolor{gray97}{gray}{.97}
\definecolor{gray75}{gray}{.75}
\definecolor{gray45}{gray}{.45}

\usepackage{listings}
\lstset{ frame=Ltb,
	framerule=0pt,
	aboveskip=0.5cm,
	framextopmargin=2pt,
	framexbottommargin=2pt,
	framexleftmargin=0.4cm,
	framesep=0pt,
	rulesep=.4pt,
	backgroundcolor=\color{white},
	rulesepcolor=\color{black},
	%
	stringstyle=\ttfamily,
	showstringspaces = false,
	basicstyle=\small\ttfamily,
	commentstyle=\color{gray45},
	keywordstyle=\bfseries,
	%
	numbers=none,
	numbersep=15pt,
	numberstyle=\tiny,
	numberfirstline = false,
	breaklines=true,
}

\usepackage{listings}
\lstset{ frame=Ltb,
	framerule=0pt,
	aboveskip=0.5cm,
	framextopmargin=2pt,
	framexbottommargin=2pt,
	framexleftmargin=0.4cm,
	framesep=0pt,
	rulesep=.4pt,
	backgroundcolor=\color{white},
	rulesepcolor=\color{black},
	%
	stringstyle=\ttfamily,
	showstringspaces = false,
	basicstyle=\small\ttfamily,
	commentstyle=\color{gray45},
	keywordstyle=\bfseries,
	%
}

% minimizar fragmentado de listados
\lstnewenvironment{listing}[1][]
{\lstset{#1}\pagebreak[0]}{\pagebreak[0]}

\lstdefinestyle{consola}
{basicstyle=\scriptsize\bf\ttfamily}

\lstdefinestyle{}{language=c++,}

\begin{document}

\title{\bf \LARGE Moogle!}
\author{Facultad de Matem\'atica y Computaci\'on \\ Universidad de La Habana \\ \\ Eduardo Brito Labrada}
\date{\today}

\maketitle

\section*{Una breve introducci\'on}

Moogle! es una aplicación {\it totalmente original} cuyo propósito es buscar inteligentemente un texto en un conjunto de documentos. Es una aplicación web, desarrollada con tecnología {\tt \color{gray45}.NET Core 6.0}, específicamente usando Blazor como {\it framework} web para la interfaz gráfica, y en el lenguaje {\tt \color{gray45} C\#}.
La aplicación está dividida en dos componentes fundamentales:

\begin{itemize}
	\item {\tt \color{gray45} MoogleServer} es un servidor web que renderiza la interfaz gráfica y sirve los resultados.
	\item {\tt \color{gray45} MoogleEngine} es una biblioteca de clases donde está\dots ehem\dots casi implementada la lógica del algoritmo de búsqueda.
\end{itemize}

\subsection*{?`Para qu\'e sirve?}

La idea original del proyecto es buscar en un conjunto de archivos de texto (con extensión {\tt \color{gray45} .txt}) que estén en la carpeta {\tt \color{gray45}Content}.

\subsection*{?`C\'omo usarlo?}

Primeramente, se aconseja a quien use esta aplicaci\'on tener instalado \href{https://es.wikipedia.org/wiki/Linux}{Linux}, ya que no se garantiza la misma eficiencia si esta en un dispositivo que use Windows.
En caso de tener instalado Windows, puede optar por \href{https://learn.microsoft.com/es-es/windows/wsl/install}{instalar Windows Subsystem for Linux (WSL)} que a\~nade
funcionalidades de Linux en Windows.

\subsubsection*{Instrucciones}

Lo primero que tendrás que hacer para poder trabajar en este proyecto es \href{https://learn.microsoft.com/es-es/dotnet/core/install/}{instalar {\tt \color{gray45}.NET Core 6.0}}.
Luego, te debes parar en la carpeta del proyecto y dependiendo de tu Sistema Operativo hacer lo siguiente:

\begin{itemize}
	\item {\bf Linux:} Debes tener instalado {\tt \color{gray45} make}. Si no lo tienes instalado
puedes instalarlo usando {\tt \color{gray45}sudo apt update \&\& sudo apt install make}. Luego podr\'as hacer {\tt \color{gray45}make dev}
	
	\item {\bf Windows:} Deberías poder ejecutar este proyecto usando {\tt \color{gray45}dotnet watch run --project MoogleServer}
\end{itemize}

Despu\'es de hacer lo anterior abre en tu navegador \href{http://localhost:5000}{http://localhost:5000} y podr\'as usar Moogle! introduciendo tu b\'usqueda en la ``entrada'' y luego presionando el bot\'on ``Buscar''.

\section*{Motor de b\'usqueda}

El motor de búsqueda usa un modelo vectorial que nos computa para una {\it query} dada 
qué tan relevante es un documento determinado. Este modelo vectorial usa  {\it Term Frequency and Inverse Document Frequency (TF-IDF)} con 
{\it Cosine Similarity} para computar la relevancia. 

Para computar el vector {\it TF-IDF} uso la fórmula:

\[ TFIDF = (\frac{tf}{tw}) \times \ln(\frac{td}{dt}) \]

En esta fórmula tenemos que:
\begin{itemize}
 \item $tf$ es la frecuencia del término en el documento actual.
 \item $tw$ es la cantidad de palabras totales en el documento actual.
 \item $td$ es la cantidad total de documentos a analizar.
 \item $dt$ es la cantidad de documentos que contienen el término.
\end{itemize}

{\bf \color{red} Nota importante}: dado que $\frac{td}{dt}$ puede causar problemas por la 
división entre $0$ decidí que si $dt = 0$ luego $TFIDF = 0$, tiene sentido hacer esto ya 
que si $dt = 0$ el término no aparece en ningún documento.\\

Después de hacer lo anterior necesitamos calcular la ``similitud'' entre el vector de 
{\it document} y el vector de la {\it query}, que para eso utilizo 
{\it Cosine Similarity}. 
Para hacer esto intentamos estimar el ``ángulo'' comprendido entre el vector de la 
{\it query} y el vector de {\it document}: mientras menor sea este ángulo, mayor 
``similitud'' tendrán estos vectores. Esto lo hacemos usando la fórmula:

\[\cos \alpha = \frac{v_d \cdot v_q}{||v_d|| ~ ||v_q||}\]

Siendo:

\begin{itemize}
 \item $v_d$ el vector de {\it document}
 \item $v_q$ el vector de {\it query}
 \item $||v||$ es la magnitud del vector $v$
\end{itemize}

\section*{Sobre la implementaci\'on}

\subsection*{\tt Utils.cs}

Esta clase est\'a principalmente para m\'etodos que son necesarios en varias clases del
proyecto.

\begin{itemize}
	\item {\tt int EditDistance(string a, string b)}:  devuelve el menor n\'umero de operaciones de edici\'on
que se deben hacer para igualar ambas cadenas (las operaciones son insertar un caracter, eliminar un caracter, 
cambiar un caracter por otro). Para que esto funcione uso un algoritmo de Programaci\'on Din\'amica con una 
optimizaci\'on en memoria para que en lugar de tomar $O(n \times m)$ en memoria tome $O(2 \times min(n, m))$, 
a pesar de esto la complejidad temporal sigue siendo la misma $O(n \times m)$, donde $n$ y $m$ son las longitudes de las cadenas.
	\item {\tt int LongestCommonPrefix(string a, string b)}: esta clase recibe dos cadenas y
devuelve el prefijo com\'un m\'as largo de dichas cadenas. Complejidad temporal $O(min(n, m))$ donde $n$ y $m$ 
son las longitudes de las cadenas.
	\item {\tt double Distance(string a, string b)}: devuelve la similitud de dos cadenas usando el {\tt Edit Distance}, pero
esto no siempre funciona como se espera, por ejemplo {\tt ebelabrada} est\'a tan cerca a {\tt eblabrada} como a {\tt zbelabrada}, 
sin embargo deber\'ia devolver a {\tt eblabrada} que tiene un mayor prefijo com\'un. Para arreglar esto utilizo 
$\frac{ed}{lcp}$ donde $ed$ es el resultado de {\tt EditDistance} y $lcp$ es el resultado de {\tt LongestCommonPrefix}.
	\item {\tt bool AreSimilar(string a, string b)}: devuelve {\tt true} si dos palabras son similares, considero dos palabras
similares si su distancia de edici\'on es a lo m\'as $1$.
	\item {\tt string Capitalize(string word)}: devuelve a {\tt word} capitalizado.
	\item {\tt string Tokenizer(string word)}: devuelve a {\tt word} eliminando todos los caracteres que no sean letras o digitos.
	\item {\tt List<string> NormalizeText(string text)}: dado un texto devuelve una lista de todas las palabras dentro del texto, 
	adem\'as las palabras en esta lista ser\'an devueltas "tokenizadas".
	\item {\tt string[] GetNeed(string[] words)}: devuelve un {\tt Array} de las palabras que tienen a {\tt \^{}} delante.
	\item {\tt string[] GetForbidden(string[] words)}: devuelve un {\tt Array} de las palabras que tienen a {\tt !} delante.
	\item {\tt (string, int)[] GetMore(string[] words)}: devuelve un {\tt Array} de tuplas, en el primer ``item'' de la tupla 
	est\'a la palabra que contiene {\tt *} delante y en el segundo ``item'' de la tupla contiene la cantidad de {\tt *} delante 
	que contiene la palabra.
	\item {\tt (string, string)[] GetNear(string[] words)}: devuelve un {\tt Array} de tuplas que contiene las palabras que est\'an relacionadas por 
	{\tt \~{}}
	\item {\tt double Norm(Dictionary<string, double> vec)}: devuelve la norma de un vector.
\end{itemize}

\subsection*{\tt TFIDFAnalyzer.cs}

Esta es la clase m\'as importante. En ella calculo el TF, el IDF de todas las palabras. 
Adem\'as de la relevancia de una palabra en un documento determinado.

\begin{itemize}
	\item {\tt TFIDFAnalyzer(string path)}: dado un {\tt path} el constructor calcula el TF 
	de cada palabra en cada documento, el IDF de todas las palabras, la relevancia de cada 
	palabra en cada documento. Adem\'as, aqu\'i se guarda la informaci\'on calculada en los 
	{\tt .json} que aparecen en la carpeta {\tt Database}, o se toma la informaci\'on guardada 
	si ya esta calculada, de esta forma se evitan hacer los c\'alculos dos veces.
	\item {\tt bool CanGet(string database = "../Database")}: devuelve {\tt true} si la cantidad
de archivos que hay guardados en la carpeta {\tt Database} es la necesaria para tener {\bf toda} 
la informaci\'on de {\bf todos} los documentos en {\tt Content}. Esto significa que si se modifica/elimina/a\~nade 
al\'gun documento no va a devolver {\tt true}, o si no est\'an los {.json} necesarios
para recuperar la informaci\'on tampoco devolver\'a {\tt true}.
	\item {\tt void SaveInfo(string database = "../Database")}: se encarga de guardar toda la informaci\'on en los {\tt .json} dentro de la carpeta {\tt Database}.
	\item {\tt void GetInfo(string database = "../Database")}: se encarga de obtener toda la informaci\'on guardada en los {\tt .json} que aparecen en la carpeta {\tt Database}.
	\item {\tt void DeleteInfo(string database = "../Database")}: se encarga de eliminar los {\tt .json} que aparecen en la carpeta {\tt Database}.
	\item {\tt void ProcessDocuments(List<string> doc, int index)}: es un m\'etodo auxiliar que dado el documento como una lista de palabras devuelve calcula el TF de cada una de esas
palabras dentro del documento dado. {\tt index} es el indice de este documento.
	\item {\tt string Suggestion(string query)}: dada una {\it query} devuelve una sugerencia de b\'usqueda para esta {\it query}. Esta sugerencia es calculada 
	usando el {\tt EditDistance}, por cada palabra de la {\it query} busca la palabra dentro del vocabulario que tenga menor distancia de edici\'on con ella.
	\item {\tt double OperatorIn(string[] words, int index)}: este m\'etodo es utilizado para el operador {\tt \^{}} devuelve $1.0$ si todas las palabras en {\tt words} 
	aparecen en el documento con \'indice {\tt index}, de lo contrario devuelve $0.0$.
	\item {\tt double OperatorNotIn(string[] words, int index)}: este m\'etodo es utilizado para el operador {\tt !} devuelve $1.0$ si ninguna de las palabras en {\tt words} 
	aparece en el documento con \'indice {\tt index}, de lo contrario devuelve $0.0$.
	\item {\tt double OperatorMore((string, int)[] words, int index)}: dado las palabras que contienen {\tt *} y la cantidad de veces que aparece el {\tt *} delante, devuelve $\prod more * \ln(freq_{words})$ donde $more$ es la cantidad de veces que {\tt *} aparece delante de la palabra y $freq_{word}$ es la cantidad de veces que $word$ aparece en el documento con \'indice {\tt index}.
	\item {\tt double OperatorNear((string, string)[] words, int index)}: por cada par de palabras en {\tt words} calcula la m\'inima distancia entre las apariciones de estas palabras en el documento con \'indice {\tt index}, esa m\'inima distancia (llamemosle $md$), la utilizo para calcular lo que le aporta a la respuesta, $\prod \frac{2}{\ln(md) + 1}$.
	\item {\tt double ComputeRelevance(Dictionary<string,double> queryVec, int index, string[] need, string[] forb, (string, int)[] more, (string, string)[] near)} este m\'etodo recibe el vector de la {\it query}, el \'indice del documento en el que se va a analizar la query, las palabras que contienen {\tt \^{}}, {\tt !}, {\tt *} delante, y las que est\'an asociadas por {\tt \~{}}. Devuelve la relevancia de la query dentro del documento, utilizando la f\'ormula mencionada en {\bf Motor de b\'usqueda}, ese valor multiplicado por el resultado de los \'ultimos cuatro m\'etodos explicados anteriormente.
\end{itemize}

\end{document}
